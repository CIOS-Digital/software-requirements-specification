\documentclass[12pt, letterpaper]{article}
\usepackage[margin=1in]{geometry}

\usepackage{bookmark}

\usepackage{enumitem}

% References
\usepackage{hyperref}

% Import citations for use with BibTeX
\usepackage{cite}

% Table formatting
\usepackage{tabularx}

% Paragraph-blocks, allowing multi-line cells in tables.
\usepackage{makecell}

% Restrict table of contents to parts through subsubsections
\setcounter{tocdepth}{3}

\hypersetup{
    colorlinks,
    citecolor=black,
    filecolor=black,
    linkcolor=black,
    urlcolor=black
}
\title{SRS \\
Flight Planning
}
\author{ Cedrick Cooke
    \and Ian Littke
    \and Owen Roth-Lerner
    \and Sander Scherman Garzon
    \and Justine O'Neil
}
\date{Version 0.2 \\ 2017, February 23}

\begin{document}
\maketitle

\tableofcontents
\raggedright
\section*{Summary of Changes}
\begin{tabularx}{\textwidth}{|l|r|X|l|}
\hline
Editor & Revision & Description & Date \\ \hline \hline
Ian Littke & 0.1 & Initial document blocking & 2017-02-21 \\ \hline
\makecell[cl]{Cedrick Cooke,\\ Owen Roth-Lerner,\\ Sander Scherman Garzon} & 0.2 & Expanded sections 3 and 4 & 2017-02-23 \\ \hline
\end{tabularx}

\section{Introduction}
  \subsection{Purpose}
    This SRS describes the software functional and nonfunctional requirements
    for release 1.0 of the Flight Plan Editor.
    This document is intended to be used by the members of the project team that will implement
    and verify the correct functioning of the system.
    Unless otherwise noted, all requirements specified here are high priority
    and committed for release 1.0.
  \subsection{Document Conventions}
  This document features some terminology which readers may be unfamiliar with.
  See \hyperref[sec:glossary]{Appendix \ref{sec:glossary} (Glossary)} for a list of these terms and their definitions.
  The format of this SRS is simple. Bold face and indentation is used on general topics and or specific points of interest.
  The remainder of the document will be written using the standard font in \LaTeX.
  \subsection{Intended Audience and Reading Suggestions}
  This document is intended for all individuals participating in the initial release of Flight Plan Editor.
  Including, but not limited to, all project team members; Aran Clauson, team mentor; and Martin Kjelstad, CAP contact point.

  \subsection{Project Scope}
  The Flight Planning Editor will allow flight planners in the Civilian Air Patrol (CAP)
  to plan their flights in an easy fashion and store them in a Secure Digital (SD) card.
  A detailed project description is available in the \hyperref[sec:ref]{\textit{Flight Plan Editor Vision and Scope Document}}.
  The section in that document titled "Scope of Initial and Subsequent Releases" lists the features that are
  scheduled for full or partial implementation in this release.
  \subsection{References}\label{sec:ref}
  \begin{tabularx}{\textwidth}{l|X}
    \hline
    Vision and Scope & \url{https://github.com/cedrickcooke/flight-planning/tree/master/vision-scope}\\
    FAA VFR Raster Charts & \url{https://www.faa.gov/air_traffic/flight_info/aeronav/digital_products/vfr/} \\
    \hline
  \end{tabularx}

\section{Overall Description}
  This section gives an overview of the functionality of the product.
  It describes the informal requirements and is used to establish a context for the technical
  requirements specification in the next section.
  \subsection{Product Perspective}
    This flight planning editor addresses a need of the CAP in regards to emergency flight planning.
    Specifically, this is new software which allows a user to easily input latitude and longitude coordinates to develop flight plans.
    These flight plans are then able to exported onto SD cards in a format that the Garming G1000 understands.
  \subsection{Product Features}
    The most important feature of this product is the ability to read, edit, and write XML containing flight plans.
    This is done through a user interface which displays a map of the flight plan.
  \subsection{User Classes and Characteristics}
    The user is expected to have knowledge of flying area and can read charts.
    The user shall also be expected to have lattitude-longitude coordinates of
    waypoints of interest to include in the flight plan.
  \subsection{Operating Environment}
    \begin{tabularx}{\textwidth}{|l|X|} \hline
      OE-1: & The Flight Plan Editor shall operate with the following operating systems:
              Microsoft Windows 7, Microsoft Windows 8 and Microsoft Windows 10. \\ \hline
    \end{tabularx}
  \subsection{Design and Implementation Constraints}
    \begin{tabularx}{\textwidth}{|l|X|} \hline
      CO-1: & The system's design and code shall be written in C\#. \\ \hline
      CO-2: & All flight plans shall be recorded in XML. \\ \hline
    \end{tabularx}
  \subsection{User Documentation}
    \begin{tabularx}{\textwidth}{|l|X|} \hline
      UD-1: & The system shall provide hierarchiacal and cross-linked help system in HTML that
              describes and illustrates all system functions.\\ \hline
      UD-2: & The system shall also offer tooltips to assist in user functions \\ \hline
    \end{tabularx}
  \subsection{Assumptions and Dependencies}
    \begin{tabularx}{\textwidth}{|l|X|} \hline
      AS-1: & Flight operators can map around restricted areas/altitude. \\ \hline
      AS-2: & Sectional map(s) are located in the proper location. \\ \hline
      DE-1: & Windows operating system, .NET stack. \\ \hline
      DE-2: & SD Card supplied is of a certain make/model. \\ \hline
      DE-3: & Flight plan is used in Garmin G1000. \\ \hline
    \end{tabularx}

\section{System Features}
  \subsection{Display Sectional Map}
    \subsubsection{Description and Priority}
      The application will display the sectional map as the main interface.
      The map will respond to input from lattitude/longitude coordinate input as well
      as having the ability to pan around the map. Priority = high.
    \subsubsection{Stimulus/Response Sequences}
      \begin{description}
        \item[Stimulus] A user can request to create a new waypoint.
		\item[Response]	The sectional map will draw a new point on the map to
			reflect the changes made.
		\item[Stimulus] A user can also click on the map and drag their mouse 
			across the screen.
		\item[Response] The map will maneuver and scroll vertically and horizontally
			in the opposite direction of the mouse pointers movement.
		\item[Stimulus] A user may also scroll with the mousewheel.
		\item[Response] The map will either zoom in as the mousewheel is scrolling upwards.
			or the map can zoom out if scrolling downwards.
      \end{description}
    \subsubsection{Functional Requirements}
    \begin{tabularx}{\textwidth}{|l|X|} \hline
      FR-1: & Upon starting the program, the sectional map will be displayed\\ \hline
      FR-2: & This map will have the ability to maneuver and pan based upon user mouse movement\\ \hline
      FR-3: & Map will also respond to changes in waypoints, or selecting different flight plans\\ \hline
      \end{tabularx}

  \subsection{Input Destination Coordinates}
	
    \subsubsection{Description and Priority}
		In the top left corner of the application, there are two input windows for 
		lattitude/longitude coordinate input. Once the user enters the data, 
		points on the sectional map will appear as well as a text list for each 
		coordinate. Priority = high.
    \subsubsection{Stimulus/Response Sequences}
      \begin{description}
        \item[Stimulus] User will click on the lattitude text box to input the 
			lattitude, and complete the same for longitude.
		\item[Response] Upon completion, the user will hit the enter key on their
			keyboard, and the sectional map shall display the point respective
			to the lattitude/longitude input.
		\item[Stimulus] User can then request to edit an existing waypoint.
		\item[Response] The coordinate is displayed, and left with the option to
			either save or discard the changes.
      \end{description}
    \subsubsection{Functional Requirements}
    \begin{tabularx}{\textwidth}{|l|X|} \hline
      FR-4: & There will be a text input box for both lattitude and longtitude coordinate input\\ \hline
      FR-5: & Upon entering coordinates, a waypoint will be drawn on the sectional map to represent its location\\ \hline
      FR-6: & Selecting an already created waypoint allows the user to modify the coordinates, or discard the waypoint\\ \hline
      \end{tabularx} 

    \subsection{Distinguish Waypoint Coordinates}
      \subsubsection{Description and Priority}
		As a user inputs more lattitude/longitude coordinates, the program will
	    draw a new waypoint onto the sectional map for viewing as well 
		as having a small subtext containing the lattitude/longitude coordinates 
		for viewing to distinguish each waypoint. Priority = medium.
      \subsubsection{Stimulus/Response Sequences}
        \begin{description}
          \item[Stimulus] A user can request to view the listed waypoints on the
			sectional map.
		  \item[Response] The program will display the lattitude/longitude data
			for that given point on the sectional map, as well as highlight the
			list of waypoints that were already added to that particular flight
			plan.
        \end{description}
      \subsubsection{Functional Requirements}
      \begin{tabularx}{\textwidth}{|l|X|} \hline
        FR-7: & Each waypoint drawn on the map shall display the lattitude and longitude coordinates\\ \hline
        FR-8: & Each waypoint will also be displayed as a list between other points for connections\\ \hline
        \end{tabularx}


      \subsection{Draw Flight Path on Section Map Between Coordinates}
        \subsubsection{Description and Priority}
        		The user should have the ability to not only add or delete points, but change in which
        		order he connects one point to another. This allows the user to actually route between 
        		points rather than have to manually navigate between them. (Priority = High)
        \subsubsection{Stimulus/Response Sequences}
            Upon selection/editing a flight plan, the program will pull up a list of previous flight
            points and the path between them on a map. These points will be pathed in the order the
            appear on the list, and changing the order will change what point each point connects to.
            The user will be able to see this editing of order (and thereby flight path) automatically
            on the map.
        \subsubsection{Functional Requirements}
          \begin{tabularx}{\textwidth}{|l|X|} \hline
            FR-9: & The point list shall have ability to edit ordering of lat/long coordinates\\ \hline
            FR-10: & If adding new lat/long coordinates, the previous coord will connect to new one.\\ \hline
            FR-11: & If coordinate is last on list, the path will end there.\\ \hline
      	    FR-12: & The map shall update to show new path as the list is edited/extended.\\ \hline
          \end{tabularx}
        \subsection{Open and Save Flight Plan in XML to SD Card}
          \subsubsection{Description and Priority}
            The Garmin G1000 uses SD cards to store flight plans, and does so in the XML format.
            For this program to be of any use, it \emph{must} support reading and writing these files.
            Priority = High
          \subsubsection{Stimulus/Response Sequences}
            \paragraph{Opening a file:}
            \begin{description}
              \item[Step 1] User initiates an open action.
              \item[Step 1.1] If the user has an unsaved, open flight plan, prompt the user if they want to save it.
                If the user chooses to save the plan, complete the `saving a file' section before continuing.
              \item[Step 2] User is prompted to select a file to open.
              \item[Step 3] The file is deserialized and displayed.
              \item[Step 3.1] If the file is invalid, corrupt, or otherwise fails to open, display a warning to the user.
            \end{description}
            \paragraph{Saving a file:}
            \begin{description}
              \item[Step 1] User initiates a save action or a save-as action.
              \item[Step 2] If the user selected a save-as action, or if the current plan is not already associated with a file,
                the user selects the file location and file name they would like to use.
              \item[Step 3] The current flight plan is serialized to XML, and written to the file.
              \item[Step 4] After writing the file completes, a message is displayed to alert the user.
            \end{description}
          \subsubsection{Functional Requirements}
          \begin{tabularx}{\textwidth}{|l|X|} \hline
            FlightPlan.XML.Open & The program shall let a user open files. \\ \hline
            FlightPlan.XML.Open.Select & The program shall prompt the user to select a file to open. \\ \hline
            FlightPlan.XML.Open.Failure & After a plan is selected to open, if it fails, the user shall be alerted to the failure. \\ \hline
            FlightPlan.XML.Open.Success & After a plan is selected to open, if it succeeds, the user interface should automatically update to reflect the plan. \\ \hline
            FlightPlan.XML.Save & The program shall let a user save files. \\ \hline
            FlightPlan.XML.Save.Select & If the program cannot detect where the file was opened from,
                or if the save-as option was used to initiate the save, the program shall prompt a user for a location to save the file. \\ \hline
            FlightPlan.XML.Save.Failure & If an attempt to save a file fails, the user shall be alerted to the failure. \\ \hline
            FlightPlan.XML.Save.Success & If an attempt to save a file succeeds, the program shall display a message indicating the success. \\ \hline
          \end{tabularx}

\section{External Interface Requirements}
  \subsection{User Interfaces}
    \begin{tabularx}{\textwidth}{|l|X|} \hline
      UI-1: & The UI shall have ability to input lattitude/longitude coordinates. \\ \hline
      UI-2: & The UI shall display a chart that can be moved and arranged. \\ \hline
      UI-3: & The UI shall have a way to load and save XML flight plan data. \\ \hline
    \end{tabularx}
  \subsection{Hardware Interfaces}
    \begin{tabularx}{\textwidth}{|l|l|X|} \hline
      HI-1: & SD card writer with SD card & Location where flight plans are saved. \\ \hline
      HI-2: & Monitor, keyboard, and mouse & User interaction is optimized for a standard desktop. \\ \hline
    \end{tabularx}

  \subsection{Software Interfaces}
    \begin{tabularx}{\textwidth}{|l|l|X|} \hline
      SI-1: & Microsoft Windows 7, 8, 10 & Primary operating systems supported. \\ \hline
      SI-2: & .NET 4.5 & Runtime environment for the application. \\ \hline
      SI-3: & An XML library & A specific XML library will be identified at a later time.
        This will be used to verify the correctness of files. \\ \hline
    \end{tabularx}

  \subsection{Communications Interfaces}
  No communications interfaces exist.


  \subsection{Performance Requirements}
    \begin{tabularx}{\textwidth}{|l|X|} \hline
      PE-1: & The UI shall refactor the tiled maps within 1 minute. \\ \hline
    \end{tabularx}

  \subsection{Safety Requirments}
  No safety requirements have been identified.
  \subsection{Security Requirements}
  No security requirements have been identified.
  \subsection{Software Quality Attributes}
	No additional software quality attributes at this time.

\section{Other Requirements}
  \subsection{Key Milestones}
  \begin{tabularx}{\textwidth}{l c l}
    \hline
    \textbf{Milestone} & \textbf{Deadline} & \textbf{Comments}\\
    \hline
    \textit{SRS Document 0.1} & 2/24/2017 & Rough SRS Due \\
    \textit{Vision and Scope Document 1.0} & 4/??/2017 & Vision and Scope finalization \\
    \textit{SRS Document 1.0} & 4/??/2017 & Initial SRS Due \\
    \textit{Release 1.0} & 12/??/2017 & Release Application to CAP \\
    \hline
  \end{tabularx}
\appendix
\section{Glossary} \label{sec:glossary}
\begin{description}[style=nextline, leftmargin=10mm, topsep=0mm,noitemsep]
      \item[CAP] \hfill \ Civilian Air Patrol
      \item[FAA] \hfill \ Federal Aviation Administration
      \item[Garmin G1000] \hfill \ Integratead flight instrumentation system used in planes
      \item[GeoTIFF] \hfill \ Metadata standard which allows georeferenceing information to be embedded within a TIFF file
      \item[GPS] \hfill \ Global Positioning System
      \item[Sectional] \hfill \ Sectioned chart covering a section of an area
      \item[SRS] \hfill \ Software Requirements Specification
      \item[SD Card] \hfill \ Secure Digital Card
      \item[TIFF] \hfill \ Tagged Image File Format
      \item[VFR] \hfill \ Visual Flight Rules
      \item[XML] \hfill \ eXtensible Markup Language
  \end{description}

\section{Analysis Models}

\section{Issues List}

\bibliography{citations}{}
\bibliographystyle{plain}
\end{document}
